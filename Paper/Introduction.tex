% Introduction
\section{Introduction}

%%%%%%

% Cite:
% Wireless networks			#gupta2000capacity
% Wireless sensor networks	#akyildiz2002wireless
% Volcanic investigation		#werner2006deploying

% Mobile campus networks 	#hernandez2005comparative
% Mobile gaming community	#cunningham2002optimizing
% Energy-efficient network	#jones2001survey
% Internet of things 			#qin2014software
% Connectivity  			#moscibroda2006complexity
% Density					#wang2015connectivity
% Unit disk graph model 		#clark1990unit
% CSMA/CA				#bianchi1996performance

% The cite of the existing algorithms are listed in related work

%%%%%%

% Paper Logic Flow


Wireless sensor networks\cite{akyildiz2002wireless} are deployed in
a wide range of real-life applications today, such as volcanic investigation
\cite{werner2006deploying}, seismic detection \cite{suzuki2007high},
agriculture monitoring \cite{wang2010l3sn}, etc.
The popularity of Internet of Things (IoT) will accelerate this trend
and make wireless sensor networks even more pervasive.

In constructing a wireless sensor network,
neighbor discovery is an important fundamental step.
In most situations, a node participating in a task needs to discover its nearest
neighbors in order to carry out operations like broadcasting and peer-to-peer
communication. In this paper, we study neighbor discovery under the
constraint of limited energy. We assume a
large-scale scenario, where nodes are aware of their energy consumption and
are connected via a multi-hop network.

Despite extensive research, neighbor discovery in a large-scale network
%remains an open problem.??
remains a problem that is not fully solved.
Existing algorithms can be classified into two categories: deterministic, and probabilistic.
The algorithms presented in
Refs.~\cite{dutta2008practical, kandhalu2010u, bakht2012searchlight,
sun2014hello, chen2015heterogeneous, qiu2016talk} are %sensors take actions
based on deterministic sequences.
Most deterministic algorithms however were designed for two
nodes only, despite being then applied to a multi-node scenario.
On the other hand,
probabilistic algorithms handle neighbor discovery in a clique of $n$
nodes\cite{mcglynn2001birthday, vasudevan2009neighbor, you2011aloha,
song2014probabilistic}, i.e. every two nodes are neighbors, and utilize
the global number $n$ to compute an optimal probability for action
decisions. However, a large-scale network is generally not a clique and
node to node communication could go through multiple hops.
In addition, prominent existing
algorithms such as Birthday \cite{mcglynn2001birthday} and Aloha-like
\cite{vasudevan2009neighbor} do not consider energy consumption of the
neighbor discovery process, which could have a negative impact on the
energy state of the sensors.
%Furthermore, in wireless sensor networks,
%sensors are powered by battery and neighbor discovery will be constraint
%to energy limits.

To address the problems identified above, we looked into the existing
neighbor discovery algorithms and found that the key issue lies in the
way collisions are dealt with in large-scale networks.
%(collisions result in CSMA \cite{bianchi1996performance} to wait for more time).
%This issue is due to three reasons.???
There are three aspects to this issue.
First, transmission signals fade with distance and simultaneous
transmissions will cause collisions. %among various nodes.
Deterministic algorithms aiming at two nodes \cite{kandhalu2010u,
chen2015heterogeneous} fail to reduce such collisions. Some beacon-based
algorithms \cite{dutta2008practical, bakht2012searchlight, sun2014hello,
qiu2016talk} do not have this problem but their time slot is 40 times
larger \cite{kandhalu2010u} and therefore resulting in long latency. 
%There are many mature interference models to depict the communication collisions, and we choose the popular protocol model\cite{clark1990unit} to begin this research area.
Second, a large-scale network is not one-hop network and a node can
practically only discover those neighbors that are within its range.
Probabilistic algorithms
\cite{vasudevan2009neighbor, you2011aloha, song2014probabilistic}
assuming a small clique network fail to estimate the number of
neighbors, and thus cannot reduce the collisions effectively since the
number of neighbors provides an important hint on how many collisions
will occur at the same time.
Third, nodes have limited energy and they only have a small time window
to find their neighbors. Indeed, energy conservation and
neighbor discovery are two conflicting goals in existing algorithms.
%since collisions cause great energy consumption and if taken energy consumption into account, neighbor discovery may become even
ineffective.

We have conducted experiments to confirm the issue in real actions.
We deployed $16$ EZ$240$ sensors \cite{huang2012easipled} and
found existing algorithms to be either insufficient or
excessive in the way they deal with collisions, and
both would result in long latency.
%collisions caused by simultaneous transmissions result in the waste of time and energy. 
%We consider a large-scale network with 2000 nodes obeying a uniform distribution. 
As the number of neighbors increased, collisions of Hedis
\cite{chen2015heterogeneous} happened as frequently as 10.1\% to
19.96\%, evoking the CSMA \cite{bianchi1996performance} function in the MAC
layer to wait for a random time. Hello \cite{sun2014hello} utilized a
beacon mechanism to avoid collisions but the time slot was 40 times
larger and it resulted in 10 times longer latency. An Aloha-like method
\cite{you2011aloha} showed a high idle rate (when no neighbors are
transmitting) of 18.92\%, which reduced the collisions, but
excessively.  %and still results in a high latency.
%This is the same with CSMA \cite{bianchi1996performance}, a general collision avoidance  technique in networks, the idle rate of which is XX\% by simulations. 
All these algorithms could not achieve low latency and energy
conservation at the same time simply because they failed to deal with
collisions effectively.

Our key hypothesis is that, using an estimate of the expected number of
neighbors of a node and synchronizing the times the nodes turn on the
radio, %with its neighbors,
both low-latency and energy-efficiency in neighbor discovery can be achieved. 
We take the distribution of nodes into consideration. As studied
in \cite{wang2013gaussian}, nodes in a wireless sensor network are
likely to follow a uniform or a Gaussian distribution.
%for detecting aims. % (as shown in Fig. \ref{distribution})
According to the local density, a node can estimate the number of
neighbors and calculate an optimal probability for action decisions.
Based on this, we propose Alano, %\footnote{Alano is the god of
luck in Greek mythology.}
a nearly optimal probability based algorithm for a large-scale network.
We play on the duty cycle mechanism
\cite{zhang2017performance} which is the fraction of time the radio is on
(i.e., the sensor is woken up), and deterministically
synchronize the wake-up times between neighbors in Alano.
Specifically, if all nodes have the same (symmetric) duty cycle, such as
a fleet of sensors having a default duty cycle setting, we propose the
Relaxed Difference Set based algorithm (called RDS-Alano); if nodes have
different (asymmetric) duty cycles, such as when sensors would adjust the duty
cycle according to the remaining energy, we propose the Traversing Pointer based
algorithm (called TP-Alano).

In the simulations, we have found that Alano achieves $31.35\%$ to $ 85.25\%$
lower latency, higher discovery rate, and better scalability in large
scale networks. %, and robustness,
In comparison with the state-of-the-art algorithms \cite{you2011aloha, sun2014hello, chen2015heterogeneous, bakht2012searchlight}
Alano reaches nearly $100\%$ discovery in twice as fast a speed. 
When the number of nodes increases from 1000 to 9000, 
Alano shows 4.68 times to 6.51 times lower latency for neighbor discovery.

% Contribution summarize
The contributions of the paper are summarized as follows:
\begin{itemize}
\item[1)] We utilize the distribution of nodes and propose Alano, a
nearly optimal algorithm that can achieve low-latency neighbor discovery
for a large-scale network.
\item[2)] In an energy-restricted large-scale network, we propose
RDS-Alano for symmetric nodes and TP-Alano for asymmetric nodes. Both
algorithms achieve low latency for discovering neighbors and can prolong
the nodes' lifetime.
\item[3)] We conduct experiments for %fundamental observation
and extensive simulations for large-scale networks, in which
Alano achieves lower latency, higher discovery rate, and better scalability.
The results show Alano's promises for deployment in
%which promises a potential scalability of
IoT in the future.
%, and robustness compared with the state-of-the-art algorithms. 
\end{itemize}

%% Remaining structure
The remainder of the paper is organized as follows. The next section
highlights the related works and their unsolved problems.
Section \ref{sectionmodel} presents
the system model and basic definitions.
We introduce Alano and show the
method to combine the nodes' distribution in the design
in Section \ref{PCN}. We propose
two modified algorithms (RDS-Alano, TP-Alano) for an energy-restricted
large-scale network for both symmetric and asymmetric nodes in Section
\ref{EEN}. The simulation results are given and discussed in Section
\ref{Evaluation}, and we conclude the paper in Section \ref{Conclusion}.
