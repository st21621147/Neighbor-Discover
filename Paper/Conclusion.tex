% This section will talk about conclusion of the paper
\section{Conclusion}
\label{Conclusion}
In this paper, we systematically study the process of neighbor discovery
in an energy-restricted large-scale network. To begin with, we propose
Alano for a large-scale network where the nodes' distribution is utilized to
decide a node's transmitting probability. For different distributions,
such as the uniform distribution and the normal distribution, we show that Alano
achieves nearly optimal discovery latency. Then, we propose two modified
methods for an energy-restricted network on the basis of different duty
cycle mechanisms: Relaxed Different Set based Alano (RDS-Alano) for
symmetric nodes and Traversing Pointer based Alano (TP-Alano) for
asymmetric nodes. We conduct extensive simulations to compare Alano with
the state-of-the-art algorithms, and the results show that Alano achieves
better performance regarding discovery latency, discovery rate, and
scalability. In our future work, we will study the diversity of the nodes' 
distribution and the characteristics of a dynamic network,  
to achieve a more practical and efficient deployment in IoT.
