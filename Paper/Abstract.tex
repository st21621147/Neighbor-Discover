% As a general rule, do not put math, special symbols or citations
% in the abstract
\begin{Abstract}

Neighbor discovery is a crucial step in the construction of wireless networks. 
%Various methods have been proposed to promote the discovery rate and minimize the discovery latency. 
In our work we also consider the connectivity and the distribution of the nodes in a large-scale network.
In a large-scale network, each node has the ability to sense other nodes within its transmission range,
called neighbors. Other nodes are connected by multi-hop communication.
%Reduce abstract

In this paper, we propose Alano, the first nearly optimal probability 
based algorithm for neighbors discovery in large-scale networks. 
Initially, we examine the distribution of the nodes in the networks 
and compute the expected number of neighbors $n$ using the local information.
Then, we prove the discovery latency of Alano is bounded by a low latency $O(n ln n)$.
Finally, we propose a $Relaxed Difference Set$ based Alano algorithm (RDS-Alano) to 
achieve low-latency neighbor discovery process in the symmetric energy-efficient networks
and a $Traversing Pointer$ based Alano algorithm (TP-Alano) in the asymmetric energy-efficient networks.
Our evaluation shows Alano achieves 31.35\% to 32.32\% lower latency than existing methods 
and exhibits higher performance in quality, scalability and robustness.




\end{abstract}

% no keywords
