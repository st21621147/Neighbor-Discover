% As a general rule, do not put math, special symbols or citations
% in the abstract
\iffalse
\begin{abstract}

Neighbor discovery is a fundamental step in constructing wireless sensor networks and many algorithms have been proposed to minimize discovery latency. However, few of them can be applied to an energy-restricted large-scale network, which is more appealing and promising due to the development of intelligent devices. 
In an energy-restricted large-scale network, a node has limited power supply and can only discover nodes within its range; additionally, the discovery process may fail if surplus communication exists on the wireless channel. These factors make neighbor discovery a challenging task in establishing the networks.

In this paper, we propose Alano, a nearly optimal algorithm for a large-scale network on the basis of nodes' distributions.
When nodes have same energy constraints, we modify Alano by Relaxed Difference Set (RDS) (denote as RDS-Alano); while we present a Traversing Pointer (TP) based Alano (denote as TP-Alano) when the energy constraints are different. We compare Alano with the state-of-the-art algorithms through extensive evaluations, and the results show that Alano achieves at least $31.35\%$ lower discovery latency and it has higher performance regarding quality (discovery rate) and scalability.% and robustness.
\end{abstract}
\fi

\begin{abstract}

Neighbor discovery is a fundamental step in constructing wireless sensor
networks and many algorithms have been proposed aiming to minimize its latency.
%of the implementation.
Recent developments of intelligent devices call
for new algorithms, which are subject to energy restrictions. In
energy-restricted large-scale networks, a node has limited power supply
and can only discover other nodes that are within its range. Additionally, the
discovery process may fail if excessive communications take place in a
wireless channel. These factors make neighbor discovery a very
challenging task and only few of the proposed neighbor discovery
algorithms can be applied to energy-restricted large-scale networks. In
this paper, we propose Alano, a nearly optimal algorithm for a
large-scale network, which uses the nodes' distribution as an input.
When nodes have the same energy constraint, we modify Alano by the
Relaxed Difference Set
(RDS), and present a Traversing Pointer (TP) based Alano when
the nodes' energy constraints are different. We compare Alano
with the state-of-the-art algorithms through extensive evaluations, and
the results show that Alano achieves at least $31.35\%$ lower discovery
latency and has higher performance regarding quality (discovery rate)
and scalability.
\end{abstract}

% no keywords
