% As a general rule, do not put math, special symbols or citations
% in the abstract
\begin{abstract}

Neighbor discovery is a crucial step in constructing wireless networks. 
Various methods have been proposed to promote the discovery rate and minimize the discovery latency. 
However, none of them has considered the connectivity and the distribution of the nodes in a large-scale network.
In a large-scale network, each node has a capacity to sense the nodes within its transmission range 
which are called neighbors and other nodes are connected by multi-hop communication.

In this paper, we propose Alano, the first nearly optimal probability 
based algorithm to discover neighbors in large-scale networks. 
To begin with, we consider the distribution of the nodes in the networks 
and compute the expected number of neighbors $n$ using the local information.
Then, we prove the discovery latency of Alnano is bounded by a low latency $O(nlnn)$.
Finally, we propose a Relaxed Difference Set based Alano algorithm (RDS-Alano) to 
achieve low-latency neighbor discovery process in the symmetric energy-efficient networks
and a Traversing Pointer based Alano algorithm (TP-Alano) in the asymmetric energy-efficient networks.
Our evaluation shows Alano achieves 31.35\% to 32.32 times lower latency than existing methods 
and holds higher performance in quality, scalability and robustness.




\end{abstract}

% no keywords